% In this file you should put macros to be used only by
% the printed version. Of course they should have a corresponding
% version in macros/web.tex.
% Typically the printed version could have more fancy decorations.
% This should be a very short file.
%
% This file starts with dummy macros that ensure the pdf
% compiler will ignore macros provided by plasTeX that make
% sense only for the web version, such as dependency graph
% macros.


% Dummy macros that make sense only for web version.
\newcommand{\lean}[1]{}
\newcommand{\discussion}[1]{}
\newcommand{\leanok}{}
\newcommand{\mathlibok}{}
\newcommand{\notready}{}
% Make sure that arguments of \uses and \proves are real labels, by using invisible refs:
% latex prints a warning if the label is not defined, but nothing is shown in the pdf file.
% It uses LaTeX3 programming, this is why we use the expl3 package.
\ExplSyntaxOn
\NewDocumentCommand{\uses}{m}
 {\clist_map_inline:nn{#1}{\vphantom{\ref{##1}}}%
  \ignorespaces}
\NewDocumentCommand{\proves}{m}
 {\clist_map_inline:nn{#1}{\vphantom{\ref{##1}}}%
  \ignorespaces}
\ExplSyntaxOff

\newcommand{\seca}{\section}
\newcommand{\secb}{\subsection}
\newcommand{\secc}{\subsubsection}

\newtheorem{tactic}{Tactic}[section]

\setmonofont{Consolas}

\usepackage{titlesec}
\titleformat{\subsubsection}[runin]{\itshape}{\thesubsubsection}{1em}{}[.]

\newcommand{\leaninline}[1]{{\small \texttt{#1}}}
\newcommand{\tacticinline}[1]{\textcolor{tacticcolor}{\small \texttt{#1}}}
